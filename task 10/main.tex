\documentclass[a4paper]{article}

\usepackage[T2A]{fontenc}
\usepackage[utf8]{inputenc}
\usepackage[russian]{babel}


\usepackage{graphicx}
\usepackage{float}
\usepackage{mathtools}
\usepackage{wrapfig}
\usepackage{amsfonts, amssymb, amsmath, latexsym}
\usepackage{nicefrac}
\usepackage{hhline}
\usepackage{multirow}
\usepackage[colorinlistoftodos,bordercolor=orange,backgroundcolor=orange!20,linecolor=orange,textsize=scriptsize]{todonotes}
\usepackage[colorlinks=true,linkcolor=blue,citecolor=blue]{hyperref}       % hyperlinks
\usepackage{nicefrac}       % compact symbols for 1/2, etc.
\usepackage{nameref}
\usepackage{booktabs}       % professional-quality tables

\usepackage{algorithm}
\usepackage{algpseudocode}

\usepackage{xcolor, colortbl}
\usepackage{etoolbox}

% \graphicspath{ {./} }

\usepackage[verbose=true,letterpaper]{geometry}

\newgeometry{
    textheight=9.5in,
    textwidth=6in,
    top=1in,
    headheight=12pt,
    headsep=25pt,
    footskip=30pt
}

\usepackage{epigraph}

%

\newcommand{\argmin}{\mathop{\arg\!\min}}
\newcommand{\argmax}{\mathop{\arg\!\max}}

\newcommand{\Var}{\mathbb{V}}
\newcommand{\Exp}{\mathbb{E}}
\newcommand{\Cov}{\text{Cov}}
\newcommand{\makebold}[1]{\boldsymbol{#1}}
\newcommand{\mean}[1]{\overline{#1}}
\newcommand{\eps}{\varepsilon}
\renewcommand{\epsilon}{\varepsilon}

\newcommand{\partfrac}[2]{\frac{\partial #1}{\partial #2}}
\newcommand{\ttt}[1]{\texttt{#1}}
\newcommand{\term}[1]{\textbf{#1}}

\newcommand{\la}{\langle}
\newcommand{\ra}{\rangle}

\newcommand{\lp}{\left(}
\newcommand{\rp}{\right)}
\newcommand{\lf}{\left\{}
\newcommand{\rf}{\right\}}
\newcommand{\ls}{\left[}
\newcommand{\rs}{\right]}
\newcommand{\lv}{\left|}
\newcommand{\rv}{\right|}

\newcommand*{\affaddr}[1]{#1} % No op here. Customize it for different styles.
\newcommand*{\affmark}[1][*]{\textsuperscript{#1}}


\usepackage{subcaption}
%\usepackage[font={small}]{caption}

\usepackage{amsthm}
\usepackage{tikz}

\theoremstyle{definition}
\newtheorem{definition}{Определение}[section]

\newtheorem{exercise}{Задача}[section]

\newtheorem*{solution}{Решение}
\theoremstyle{remark}
\newtheorem*{remark}{Remark}

\makeatletter
\renewcommand{\l@section}{\@dottedtocline{1}{0em}{2.1em}}
\makeatother

% \setlength\epigraphwidth{.8\textwidth}
\setlength\epigraphrule{0pt}

\title{ТРЯП. Домашнее задание № 10}
\author{Шарапов Денис, Б05-005}
\date{}

\begin{document}

\maketitle

\section*{Задача 1}

Теорема. \textit{КС-грамматика $G = \langle N, \Sigma, P, S \rangle$ является $LL(k)$-грамматикой тогда и только тогда, когда для двух различных правил $A\rightarrow \beta$ и $A\rightarrow \gamma$ из $P$ пересечение
\begin{equation} \tag{*}
    \text{FIRST}_k(\beta\alpha) \cap~\text{FIRST}_k(\gamma\alpha)
    \
\end{equation}
пусто при всех таких $\omega A\alpha$, что $S \Rightarrow^*_l \omega A \alpha$.}

\begin{itemize}
    \item[\textbf{a})] Не является $LL(k)$-грамматикой при любом $k\in \mathbb{N}$.
    \begin{enumerate}
        \item $S \Rightarrow^*_l Aa^kb \Rightarrow_l a^{k+1}b, \;\; \omega = \varepsilon, \;\; \alpha = a^{k}b, \;\; \beta = a$,
        \item $S \Rightarrow^*_l Aa^kb \Rightarrow_l Aa^{k+1}b, \;\; \omega = \varepsilon, \;\; \alpha = a^{k}b, \;\; \gamma = Aa$.
    \end{enumerate}
    При этом пересечение (*) не пусто при любом $k\in \mathbb{N}$.
    \item[\textbf{б)}] Является $LL(k)$-грамматикой при $k > 1$.
    \begin{enumerate}
        \item $S \Rightarrow^*_l a^kAb \Rightarrow_l a^{k}\cdot a \cdot b, \;\; \omega = a^k, \;\; \alpha = b, \;\; \beta = a$,  
        \item $S \Rightarrow^*_l a^kAb \Rightarrow_l a^{k}\cdot aA \cdot b, \;\; \omega = a^k, \;\; \alpha = b, \;\; \gamma = aA$.
    \end{enumerate}
    При этом пересечение (*) пусто при $k > 1$.
    \item[\textbf{в)}] Не является $LL(k)$-грамматикой для любого $k \in \mathbb{N}$.
    \begin{enumerate}
        \item $S \Rightarrow^*_l aBBb \Rightarrow_l a\cdot ab\cdot Bb, \;\; \omega = a, \;\; \alpha = Bb, \;\; \beta = ab,$
        \item $S \Rightarrow^*_l aBBb \Rightarrow_l a \cdot \varepsilon \cdot Bb, \;\; \omega = a, \;\; \alpha = Bb, \;\; \gamma = \varepsilon.$
    \end{enumerate}
    По индукции $k$ получаем, что пересечение (*) не пусто (при $k=1, \; k=2$ берём $B\rightarrow ab$ в обоих случаях; при $k>2$ , берём $B\rightarrow \varepsilon$ в первом случае и $B \rightarrow ab$ во втором): $$\text{FIRST}_k(abBb) \cap \text{FIRST}_k(Bb) \neq \varnothing.$$
    Поэтому грамматика не является $LL(k)$ для любого $k \in \mathbb{N}$.
    \item[\textbf{г)}] Не является $LL(k)$-грамматикой для любого $k \in \mathbb{N}$.
    \begin{enumerate}
        \item $S \Rightarrow^*_l a\cdot A\cdot aAbb \Rightarrow_l a\cdot AaAb \cdot aAbb, \;\; \omega = a, \;\; \alpha = aAbb, \;\; \beta = AaAb,$
        \item $S \Rightarrow^*_l a\cdot A\cdot aAbb \Rightarrow_l a \cdot \varepsilon \cdot aAbb, \;\; \omega = a, \;\; \alpha = aAbb, \;\; \gamma = \varepsilon.$
    \end{enumerate}
    Заметим, что пересечение (*) содержит слово $w=a^k$. Поэтому грамматика не является $LL(k)$ для любого $k \in \mathbb{N}$.
    \item[\textbf{д)}] $LL(k)$-грамматика для любого $k\in\mathbb{N}$.
    \begin{enumerate}
        \item $S \Rightarrow^*_l aa \cdot B \cdot B \Rightarrow_l aa \cdot aBB \cdot B, \;\; \omega = aa, \;\; \alpha = B, \;\; \beta = aBB,$
        \item $S \Rightarrow^*_l aa \cdot B \cdot B \Rightarrow_l aa \cdot b \cdot B, \;\; \omega = aa, \;\; \alpha = B, \;\; \gamma = b.$
    \end{enumerate}
    Действительно, пересечение (*) пусто (и правило не содержит пустого слова). Поэтому эта грамматика является $LL(k)$-грамматикой.
\end{itemize} \qed 

\section*{Задача 2}

Воспользуемся алгоритмом удаления правого ветвления и получим грамматику 
$$S \rightarrow Ab, \quad A \rightarrow aA', \quad A' \rightarrow a \; | \; \varepsilon.$$

Построим таблицы $\text{FIRST}$ и $\text{FOLLOW}$:

\begin{table}[h!]
    \centering
    \begin{tabular}{|c|ccc|ccc|}
    \hline
          & \multicolumn{3}{c|}{$\text{FIRST(X)}$}                                                      & \multicolumn{3}{c|}{$\text{FOLLOW(X)}$}                                       \\ \hline
    X     & \multicolumn{1}{c|}{$S$}           & \multicolumn{1}{c|}{$A$}           & $A'$             & \multicolumn{1}{c|}{$S$} & \multicolumn{1}{c|}{$A$}           & $A'$          \\ \hline
    $F_0$ & \multicolumn{1}{c|}{$\varnothing$} & \multicolumn{1}{c|}{$\varnothing$} & $\varnothing$    & \multicolumn{1}{c|}{\$}  & \multicolumn{1}{c|}{$\varnothing$} & $\varnothing$ \\ \hline
    $F_1$ & \multicolumn{1}{c|}{$a$}           & \multicolumn{1}{c|}{$a$}           & $a, \varepsilon$ & \multicolumn{1}{c|}{\$}  & \multicolumn{1}{c|}{$b$}           & $\varnothing$ \\ \hline
    $F_2$ & \multicolumn{1}{c|}{$a$}           & \multicolumn{1}{c|}{$a$}           & $a, \varepsilon$ & \multicolumn{1}{c|}{\$}  & \multicolumn{1}{c|}{$b$}           & $b$           \\ \hline
    \end{tabular}
    \end{table}

Теперь воспользуемся алгоритмом построение анализатора:

\begin{table}[h!]
    \centering
    \begin{tabular}{|c|c|c|c|}
    \hline
         & $a$                 & $b$                          & \$    \\ \hline
    $S$  & $S \rightarrow Ab$  & error                        & error \\ \hline
    $A$  & $A \rightarrow aA'$ & error                        & error \\ \hline
    $A'$ & $A' \rightarrow a$  & $A' \rightarrow \varepsilon$ & error \\ \hline
    \end{tabular}
    \end{table}

Получившаяся грамматика является $LL(1)$-грамматикой, т. к. в каждой ячейке записано не более одного правила. \qed

\section*{Задача 3}

Для начала воспользуемся алгоритмом удаления левой рекурсии:

\begin{itemize}
    \item[] $S \rightarrow baaA \; | \; babA$,
    \item[] $A \rightarrow aA \; | \; bA \; | \; a \; | \; b$. 
\end{itemize}

Теперь удалим правое ветвление:

\begin{itemize}
    \item[] $S \rightarrow baS'$,
    \item[] $S' \rightarrow aA \; | \; bA$,
    \item[] $A \rightarrow aA \; | \; bA \; | \; \varepsilon$.  
\end{itemize}

Построим таблицу $\text{FIRST}$ и $\text{FOLLOW}$:

\begin{table}[h!]
    \centering
    \begin{tabular}{|c|ccc|ccc|}
    \hline
          & \multicolumn{3}{c|}{$\text{FIRST(X)}$}                                                        & \multicolumn{3}{c|}{$\text{FOLLOW(X)}$}                                       \\ \hline
          & \multicolumn{1}{c|}{$S$}           & \multicolumn{1}{c|}{$S'$}          & $A$                 & \multicolumn{1}{c|}{$S$} & \multicolumn{1}{c|}{$S'$}          & $A$           \\ \hline
    $F_0$ & \multicolumn{1}{c|}{$\varnothing$} & \multicolumn{1}{c|}{$\varnothing$} & $\varnothing$       & \multicolumn{1}{c|}{\$}  & \multicolumn{1}{c|}{$\varnothing$} & $\varnothing$ \\ \hline
    $F_1$ & \multicolumn{1}{c|}{$b$}           & \multicolumn{1}{c|}{$a, b$}           & $a, b, \varepsilon$ & \multicolumn{1}{c|}{\$}  & \multicolumn{1}{c|}{\$}            & $\varnothing$ \\ \hline
    $F_2$ & \multicolumn{1}{c|}{$b$}           & \multicolumn{1}{c|}{$a, b$}           & $a, b, \varepsilon$ & \multicolumn{1}{c|}{\$}  & \multicolumn{1}{c|}{\$}            & \$            \\ \hline
    \end{tabular}
    \end{table}

Построим анализатор:

\begin{table}[h!]
    \centering
    \begin{tabular}{|c|c|c|c|}
    \hline
         & $a$                 & $b$                  & \$    \\ \hline
    $S$  & error               & $S \rightarrow baS'$ & error \\ \hline
    $S'$ & $S' \rightarrow aA$ & $S' \rightarrow bA$  & error \\ \hline
    $A$  & $A \rightarrow aA$  & $A \rightarrow bA$   & $A \rightarrow \varepsilon$    \\ \hline
    \end{tabular}
    \end{table}

Работа анализатора на слове $baab$:

$$S \Rightarrow baS' \Rightarrow ba\cdot aA \Rightarrow baab.$$

\qed

\section*{Задача 5}

Заметим, что данная грамматика задаёт язык Дика, который задаётся правилами $$S \rightarrow (S)S \; | \; \varepsilon.$$ Привести исходную грамматику к такому виду можно с помощью алгоритмов удаления левой рекурсии и правого ветвления. \medskip

Докажем, что данная граматика является $LL(1)$-грамматикой.

\begin{enumerate}
    \item $S \Rightarrow^*_l (S)S \Rightarrow_l (\; (S)S \; )S \;\; \omega = (, \;\; \alpha = )S, \;\; \beta = (S)S$,  
    \item $S \Rightarrow^*_l (S)S \Rightarrow_l (\; \varepsilon \; )S, \;\; \omega = (, \;\; \alpha = )S, \;\; \gamma = \varepsilon$.
\end{enumerate}
При этом 
\begin{equation}
    \text{FIRST}_1\left[\;(S)S)S\;\right] \cap \text{FIRST}_1\left[\;)S\;\right] = \varnothing,
\end{equation}

\begin{equation}
    \text{FIRST}_1\left[S\right] \cap \text{FOLLOW}_1\left[S\right] = \varnothing.
\end{equation}

Заметим, что непосредственно после $S$ в выводах грамматики может стоять только символ~<<)>>. Поэтому верно утверждение (2). \qed

\section*{Задача 6}

Определение. Если каждое правило грамматики имеет вид либо $A \rightarrow xB$, либо $A \rightarrow x$, где $A, B \in N$, $x \in T^*$, то её называют \textit{праволинейной}. \medskip

Приведём пример праволинейной грамматики, которая не является $LL(1)$-грамматикой. Пусть $G = \langle \{S, A\}, \; \{a, b\}, \; P, \; S\rangle$, где $P$ определяется как $$S \rightarrow A, \quad A \rightarrow ab \; | \; a.$$

Она не является $LL(1)$-грамматикой по критерию (нет правила с $\varepsilon$):

$$\text{FIRST}_1(ab) \cap \text{FIRST}_1(a) = \{a\} \neq \varnothing.$$
\qed

\section*{Задача 7}

Да, верно. Рассмотрим грамматику $G = \langle \{S, A, B\}, \; \{a, b, \$\}, \; P, \; S\rangle$, где $P$ определяется как
$$S \rightarrow AB, \quad A \rightarrow \varepsilon, \quad B \rightarrow \varepsilon.$$

Построим анализатор (таблицы $\text{FIRST}$ и $\text{FOLLOW}$ содержат только $\varepsilon$ и <<\$>> соответственно):

\begin{table}[h!]
    \centering
    \begin{tabular}{|c|c|c|c|}
    \hline
        & $a$   & $b$   & \$                          \\ \hline
    $S$ & error & error & $S \rightarrow AB$          \\ \hline
    $A$ & error & error & $A \rightarrow \varepsilon$ \\ \hline
    $B$ & error & error & $B \rightarrow \varepsilon$ \\ \hline
    \end{tabular}
    \end{table}

Каждой ячейке анализатора соответствует не более одного правила, следовательно, эта граматика является $LL(1)$-грамматикой. \qed

\section*{Задача 8}

Нет, неверно. Рассмотрим грамматику $G = \langle \{S, A, B\}, \; \{a, b, \$\}, \; P, \; S\rangle$, где $P$ определяется как
$$S \rightarrow AB, \quad A \rightarrow \varepsilon, \quad B \rightarrow \varepsilon.$$

Заметим, что $\text{FOLLOW}(A) \cap \text{FOLLOW}(B) = \{\$\}$. При этом грамматика $G$ является $LL(1)$-грамматикой (задача 7). \qed

\end{document}
