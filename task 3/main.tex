\documentclass[a4paper]{article}

\usepackage[T2A]{fontenc}
\usepackage[utf8]{inputenc}
\usepackage[russian]{babel}


\usepackage{graphicx}
\usepackage{float}
\usepackage{mathtools}
\usepackage{wrapfig}
\usepackage{amsfonts, amssymb, amsmath, latexsym}
\usepackage{nicefrac}
\usepackage{hhline}
\usepackage{multirow}
\usepackage[colorinlistoftodos,bordercolor=orange,backgroundcolor=orange!20,linecolor=orange,textsize=scriptsize]{todonotes}
\usepackage[colorlinks=true,linkcolor=blue,citecolor=blue]{hyperref}       % hyperlinks
\usepackage{nicefrac}       % compact symbols for 1/2, etc.
\usepackage{nameref}
\usepackage{booktabs}       % professional-quality tables

\usepackage{algorithm}
\usepackage{algpseudocode}

\usepackage{xcolor, colortbl}
\usepackage{etoolbox}

% \graphicspath{ {./} }

\usepackage[verbose=true,letterpaper]{geometry}

\newgeometry{
    textheight=9.5in,
    textwidth=6in,
    top=1in,
    headheight=12pt,
    headsep=25pt,
    footskip=30pt
}

\usepackage{epigraph}

%

\newcommand{\argmin}{\mathop{\arg\!\min}}
\newcommand{\argmax}{\mathop{\arg\!\max}}

\newcommand{\Var}{\mathbb{V}}
\newcommand{\Exp}{\mathbb{E}}
\newcommand{\Cov}{\text{Cov}}
\newcommand{\makebold}[1]{\boldsymbol{#1}}
\newcommand{\mean}[1]{\overline{#1}}
\newcommand{\eps}{\varepsilon}
\renewcommand{\epsilon}{\varepsilon}

\newcommand{\partfrac}[2]{\frac{\partial #1}{\partial #2}}
\newcommand{\ttt}[1]{\texttt{#1}}
\newcommand{\term}[1]{\textbf{#1}}

\newcommand{\la}{\langle}
\newcommand{\ra}{\rangle}

\newcommand{\lp}{\left(}
\newcommand{\rp}{\right)}
\newcommand{\lf}{\left\{}
\newcommand{\rf}{\right\}}
\newcommand{\ls}{\left[}
\newcommand{\rs}{\right]}
\newcommand{\lv}{\left|}
\newcommand{\rv}{\right|}

\newcommand*{\affaddr}[1]{#1} % No op here. Customize it for different styles.
\newcommand*{\affmark}[1][*]{\textsuperscript{#1}}


\usepackage{subcaption}
%\usepackage[font={small}]{caption}

\usepackage{amsthm}
\usepackage{tikz}

\theoremstyle{definition}
\newtheorem{definition}{Определение}[section]

\newtheorem{exercise}{Задача}[section]

\newtheorem*{solution}{Решение}
\theoremstyle{remark}
\newtheorem*{remark}{Remark}

\makeatletter
\renewcommand{\l@section}{\@dottedtocline{1}{0em}{2.1em}}
\makeatother

% \setlength\epigraphwidth{.8\textwidth}
\setlength\epigraphrule{0pt}

\title{ТРЯП. Домашнее задание № 3}
\author{Шарапов Денис, Б05-005}
\date{}

\begin{document}

\maketitle

\section*{Задача 1}

Заменим в конструкции произведения для пересечения регулярных языков последний пункт на $$F_{\mathcal{C}} = F_{\mathcal{A}} \times Q_{\mathcal{B}} \cup Q_{\mathcal{A}} \times F_{\mathcal{B}}.$$ Верно ли, что автомат $\mathcal{C}$ распознает объединение $L(\mathcal{A}) \cup L(\mathcal{B})$?

\bigskip

\noindent \textbf{Решение.} \bigskip

Нет, неверно. Рассмотрим автоматы $\mathcal{A}$ и $\mathcal{B}$. \\

Автомат $\mathcal{A}$:

\begin{center}
\begin{tikzpicture}[scale=0.2]
\tikzstyle{every node}+=[inner sep=0pt]
\draw [black] (6.2,-7) circle (3);
\draw (6.2,-7) node {$q_0^{\mathcal{A}}$};
\draw [black] (18.8,-7) circle (3);
\draw (18.8,-7) node {$q_1^{\mathcal{A}}$};
\draw [black] (18.8,-7) circle (2.4);
\draw [black] (0.2,-7) -- (3.2,-7);
\fill [black] (3.2,-7) -- (2.4,-6.5) -- (2.4,-7.5);
\draw [black] (9.2,-7) -- (15.8,-7);
\fill [black] (15.8,-7) -- (15,-6.5) -- (15,-7.5);
\draw (12.5,-6.5) node [above] {$a$};
\draw [black] (19.845,-4.2) arc (187.2643:-100.7357:2.25);
\draw (24.25,-0.91) node [right] {$a,b$};
\fill [black] (21.66,-6.13) -- (22.51,-6.52) -- (22.39,-5.53);
\end{tikzpicture}
\end{center}

Автомат $\mathcal{B}$:

\begin{center}
\begin{tikzpicture}[scale=0.2]
\tikzstyle{every node}+=[inner sep=0pt]
\draw [black] (6.2,-7) circle (3);
\draw (6.2,-7) node {$q_0^{\mathcal{B}}$};
\draw [black] (18.8,-7) circle (3);
\draw (18.8,-7) node {$q_1^{\mathcal{B}}$};
\draw [black] (18.8,-7) circle (2.4);
\draw [black] (0.2,-7) -- (3.2,-7);
\fill [black] (3.2,-7) -- (2.4,-6.5) -- (2.4,-7.5);
\draw [black] (9.2,-7) -- (15.8,-7);
\fill [black] (15.8,-7) -- (15,-6.5) -- (15,-7.5);
\draw (12.5,-6.5) node [above] {$b$};
\draw [black] (19.845,-4.2) arc (187.2643:-100.7357:2.25);
\draw (24.25,-0.91) node [right] {$a,b$};
\fill [black] (21.66,-6.13) -- (22.51,-6.52) -- (22.39,-5.53);
\end{tikzpicture}
\end{center}

Теперь построим автомат $C$ по заданным правилам: \\

Автомат $\mathcal{C}$:

\begin{center}
\begin{tikzpicture}[scale=0.2]
\tikzstyle{every node}+=[inner sep=0pt]
\draw [black] (6.2,-4) circle (3.8);
\draw (6.2,-4) node {$q_0^{\mathcal{A}},\mbox{ }q_0^{\mathcal{B}}$};
\draw [black] (19.9,-4) circle (3.8);
\draw (19.9,-4) node {$q_0^{\mathcal{A}},\mbox{ }q_1^{\mathcal{B}}$};
\draw [black] (19.9,-4) circle (3.2);
\draw [black] (6.2,-14.7) circle (3.8);
\draw (6.2,-14.7) node {$q_1^{\mathcal{A}},\mbox{ }q_0^{\mathcal{B}}$};
\draw [black] (6.2,-14.7) circle (3.2);
\draw [black] (19.9,-14.7) circle (3.8);
\draw (19.9,-14.7) node {$q_1^{\mathcal{A}},\mbox{ }q_1^{\mathcal{B}}$};
\draw [black] (19.9,-14.7) circle (3.2);
\draw [black] (0.2,-4) -- (2.4,-4);
\fill [black] (2.4,-4) -- (1.6,-3.5) -- (1.6,-4.5);
\draw [black] (22.304,-11.776) arc (168.30085:-119.69915:2.85);
\draw (28.41,-10.17) node [right] {$a,b$};
\fill [black] (23.68,-14.83) -- (24.36,-15.48) -- (24.57,-14.5);
\draw [black] (17.371,-11.945) arc (-155.10905:-204.89095:6.165);
\fill [black] (17.37,-11.94) -- (17.49,-11.01) -- (16.58,-11.43);
\draw (16.3,-9.35) node [left] {$a$};
\draw [black] (17.176,-17.3) arc (-59.66314:-120.33686:8.169);
\fill [black] (17.18,-17.3) -- (16.23,-17.27) -- (16.74,-18.14);
\draw (13.05,-18.92) node [below] {$b$};
\end{tikzpicture}
\end{center}

Видно, что слово $w = ab$, распознаваемое автоматом $\mathcal{A}$, не распознается автоматом $\mathcal{C}$.

    \bigskip
    
\noindent  \textbf{Ответ:} Нет, не верно.

\section*{Задача 2}

Обозначим через $S_w$ язык слов с суффиксом $w$. Докажите или опровергните следующие утверждения: 

\begin{enumerate}
    \item ДКА, распознающий язык $S_w$, имеет не менее $|w| + 1$ состояний.
    \item Для каждого $w$ существует ДКА с $|w|+1$ состоянием, распознающий язык $S_w$.
\end{enumerate}

\noindent \textbf{Решение.}

\begin{enumerate}
    \item Утверждение верно. Докажем его по индукции числа $n$ букв в суффиксе $w$.
    
    База индукции $n\geq 1$: представим слово $v \in S_w$ в виде $v = xw$, где $x$ --- некоторое слово. 
    
    Для данного случая потребуется автомат с двумя состояниями. Действительно, eсли автомат не может распознать слово $w$ одним единственным состоянием, то он не сможет распознать и слово $v$ этим состоянием. Но не всякое слово длины 1 ($w$) можно распознать автоматом c одним состоянием: например, автомат ничего не принимает, либо принимает пустое слово и ещё некоторые слова, если есть переход в это состояние, и оно является принимающим. Таким образом, автомат с одним состоянием не сможет принять все слова $v \in S_w$. Откуда следует, что состояний не менее $|w|+1 = 2$.
    
    Предположение индукции $n=k-1$: пусть слово с суффиксом $w$ длины $|w|=k-1$ распознается автоматом, который имеет не менее $k$ состояний.
    
    Переход индукции $n=k$: покажем, что для распознания слова с суффиксом $w = w_1w_2\ldots w_{k}$ необходимо ещё одно состояние.
    
    В автомате для распознавания слова $w = w_1w_2\ldots w_{k-1}$ единственным принимающим состоянием является состояние под номером $k$ (иначе принимались бы слова, являющиеся префиксами слова $w$, и тогда принимались бы не все слова из $S_w$). Тогда переход из состояния $n$ возможен только в новое состояние, которое должно быть принимающим (конструктивно это выглядит так: берётся последняя буква и по ней берётся переход в последнее состояние, которое является принимающим). Отсюда следует, что необходимо ещё одно состояние. \qed
    
    \item Утверждение верно. Приведём конструктивное доказательство.
    
    
\end{enumerate}


\section*{Задача 3}

Зафиксируем последовательность языков $R_i$ над алфавитом $\Sigma = \{a,b\}$, состоящих из слов, в которых на $i$-ом месте от конца стоит $a$. Докажите, что любой ДКА $\mathcal{A}_n$, распознающий язык $R_n$ имеет не менее $2^n$ состояний.

\bigskip

\noindent \textbf{Решение.} \bigskip

\noindent От противного. Пусть существует ДКА $\mathcal{A}_n$ меньшего размера. Пусть в результате обработки произвольного слова $w$ автомат $\mathcal{A}_n$ окажется в некотором состоянии $q_w$ (слово $wab^{n-1}$ должно было быть принято). По принципу Дирихле среди всех слов длины $n$ найдётся хотя бы два различных слова, для которых $q_u = q_v$. \\

\noindent Слова различаются, следовательно, $$\exists i : u[i] \neq v[i].$$ Без ограничения общности пусть $u[i] = a$, $v[i] = b$. Автомат $\mathcal{A}_n$ принимает слово $ub^{n-i+1}$, т.~к. оно принадлежит языку $R_n$. Тогда существует последовательность конфигураций: $$(q_u, \sigma) \vdash^* (q_f, \epsilon),$$ где $q_f \in F_{\mathcal{A}_n}$. \\

\noindent Но при этом автомат $\mathcal{A}_n$ принимает слово $vb^{n-i+1}$, т.~к. существует последовательность конфигураций: $$(q_v, \sigma) \vdash^* (q_f, \epsilon).$$ Но в этом слове на $n$-ом месте от конца стоит символ $b$, поэтому $vb^{n-i+1} \notin R_n$. Противоречие. \qed

\end{document}
