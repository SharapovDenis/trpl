\documentclass[a4paper]{article}

\usepackage[T2A]{fontenc}
\usepackage[utf8]{inputenc}
\usepackage[russian]{babel}


\usepackage{graphicx}
\usepackage{float}
\usepackage{mathtools}
\usepackage{wrapfig}
\usepackage{amsfonts, amssymb, amsmath, latexsym}
\usepackage{nicefrac}
\usepackage{hhline}
\usepackage{multirow}
\usepackage[colorinlistoftodos,bordercolor=orange,backgroundcolor=orange!20,linecolor=orange,textsize=scriptsize]{todonotes}
\usepackage[colorlinks=true,linkcolor=blue,citecolor=blue]{hyperref}       % hyperlinks
\usepackage{nicefrac}       % compact symbols for 1/2, etc.
\usepackage{nameref}
\usepackage{booktabs}       % professional-quality tables

\usepackage{algorithm}
\usepackage{algpseudocode}

\usepackage{xcolor, colortbl}
\usepackage{etoolbox}

% \graphicspath{ {./} }

\usepackage[verbose=true,letterpaper]{geometry}

\newgeometry{
    textheight=9.5in,
    textwidth=6in,
    top=1in,
    headheight=12pt,
    headsep=25pt,
    footskip=30pt
}

\usepackage{epigraph}

%

\newcommand{\argmin}{\mathop{\arg\!\min}}
\newcommand{\argmax}{\mathop{\arg\!\max}}

\newcommand{\Var}{\mathbb{V}}
\newcommand{\Exp}{\mathbb{E}}
\newcommand{\Cov}{\text{Cov}}
\newcommand{\makebold}[1]{\boldsymbol{#1}}
\newcommand{\mean}[1]{\overline{#1}}
\newcommand{\eps}{\varepsilon}
\renewcommand{\epsilon}{\varepsilon}

\newcommand{\partfrac}[2]{\frac{\partial #1}{\partial #2}}
\newcommand{\ttt}[1]{\texttt{#1}}
\newcommand{\term}[1]{\textbf{#1}}

\newcommand{\la}{\langle}
\newcommand{\ra}{\rangle}

\newcommand{\lp}{\left(}
\newcommand{\rp}{\right)}
\newcommand{\lf}{\left\{}
\newcommand{\rf}{\right\}}
\newcommand{\ls}{\left[}
\newcommand{\rs}{\right]}
\newcommand{\lv}{\left|}
\newcommand{\rv}{\right|}

\newcommand*{\affaddr}[1]{#1} % No op here. Customize it for different styles.
\newcommand*{\affmark}[1][*]{\textsuperscript{#1}}


\usepackage{subcaption}
%\usepackage[font={small}]{caption}

\usepackage{amsthm}
\usepackage{tikz}

\theoremstyle{definition}
\newtheorem{definition}{Определение}[section]

\newtheorem{exercise}{Задача}[section]

\newtheorem*{solution}{Решение}
\theoremstyle{remark}
\newtheorem*{remark}{Remark}

\makeatletter
\renewcommand{\l@section}{\@dottedtocline{1}{0em}{2.1em}}
\makeatother

% \setlength\epigraphwidth{.8\textwidth}
\setlength\epigraphrule{0pt}

\title{ТРЯП. Домашнее задание № 5}
\author{Шарапов Денис, Б05-005}
\date{}

\begin{document}

\maketitle

\section*{Задача 1}

\begin{enumerate}
    \item Постройте КМП-автомат для слова $babbabab$ (на алфавитом $\{a,b\}$).
    \item Постройте для того же слова КМП-автомат $\mathcal{A}^{exc}$ с суффиксными ссылками.
    \item Продемонстрируйте работу автомата на словах:
    а) $babbabbabab$; б) $babbabc$;
\end{enumerate}

\noindent \textbf{Решение.}

\begin{enumerate}
    \item Построим КМП-автомат для слова $babbabab$.
    
    \begin{figure}[h!]
        \centering
            \begin{center}
        \includegraphics[width = 350pt]{image/picture1.png}
            \end{center}
    \end{figure}
        
    \item Построим КМП-автомат $\mathcal{A}^{exc}$ с суффиксными ссылками.
        \begin{figure}[h!]
            \centering
                \begin{center}
            \includegraphics[width = 350pt]{image/picture2.jpg}
                \end{center}
        \end{figure}
    \item Названия состояний пометим чертой сверху, чтобы отличать их от слов при использовании функции переходов.
    
    \begin{enumerate}
        \item Работа автомата $\mathcal{A}^{exc}$ на слове $w = babbabbabab, |w| = 11$: 
            \begin{eqnarray*}
                (\overline{\epsilon}, w[1,11]) \vdash (\overline{b}, w[2,11]) \vdash (\overline{ba}, w[3,11]) \vdash (\overline{bab}, w[4,11]) \vdash \\ \vdash (\overline{babb}, w[5,11]) \vdash (\overline{babba}, w[6,11]) \vdash (\overline{babbab}, w[7,11]) \vdash (\overline{babb}, w[8,11]) \vdash \\ \vdash (\overline{babba}, w[9,11]) \vdash (\overline{babbab}, w[10,11]) \vdash (\overline{babbaba}, w[11,11]) \vdash (\overline{babbabab}, \epsilon).
            \end{eqnarray*}
        \item Работа автомата $\mathcal{A}^{exc}$ на слове $w = babbabc, |w| = 7$:
            \begin{eqnarray*}
                (\overline{\epsilon}, w[1,6]c) \vdash (\overline{b}, w[2,6]c) \vdash (\overline{ba}, w[3,6]c) \vdash (\overline{bab}, w[4,6]c) \vdash \\ \vdash (\overline{babb}, w[5,6]c) \vdash (\overline{babba}, w[6,6]c) \vdash (\overline{babbab}, c) \vdash (\overline{bab}, c) \vdash (\overline{b}, c) \vdash (\overline{\epsilon}, c).
            \end{eqnarray*}
    \end{enumerate}
\end{enumerate} \qed


\section*{Задача 2}

Построить ДКА для словаря $\{ac, acb, b, ba, c, cbb\}$. Добавьте в полученный словарь слово $ab$ и удалите слово $ac$.

\bigskip

\noindent \textbf{Решение.} \bigskip

Построим ДКА для словаря $\{ac, acb, b, ba, c, cbb\}$:

\begin{center}
    \begin{tikzpicture}[scale=0.2]
    \tikzstyle{every node}+=[inner sep=0pt]
    \draw [black] (5.4,-10.6) circle (2.6);
    \draw (5.4,-10.6) node {$\epsilon$};
    \draw [black] (14.4,-2.8) circle (2.6);
    \draw (14.4,-2.8) node {$a$};
    \draw [black] (14.4,-10.6) circle (2.6);
    \draw (14.4,-10.6) node {$b$};
    \draw [black] (14.4,-10.6) circle (2);
    \draw [black] (14.4,-18.3) circle (2.6);
    \draw (14.4,-18.3) node {$c$};
    \draw [black] (14.4,-18.3) circle (2);
    \draw [black] (23.2,-2.8) circle (2.6);
    \draw (23.2,-2.8) node {$ac$};
    \draw [black] (23.2,-2.8) circle (2);
    \draw [black] (31.7,-2.8) circle (2.6);
    \draw (31.7,-2.8) node {$acb$};
    \draw [black] (31.7,-2.8) circle (2);
    \draw [black] (23.2,-10.6) circle (2.6);
    \draw (23.2,-10.6) node {$ba$};
    \draw [black] (23.2,-10.6) circle (2);
    \draw [black] (23.2,-18.3) circle (2.6);
    \draw (23.2,-18.3) node {$cb$};
    \draw [black] (31.7,-18.3) circle (2.6);
    \draw (31.7,-18.3) node {$cbb$};
    \draw [black] (31.7,-18.3) circle (2);
    \draw [black] (0.2,-10.6) -- (2.8,-10.6);
    \fill [black] (2.8,-10.6) -- (2,-10.1) -- (2,-11.1);
    \draw [black] (6.371,-8.195) arc (150.99991:110.82886:10.617);
    \fill [black] (11.88,-3.42) -- (10.96,-3.24) -- (11.31,-4.17);
    \draw (7.75,-4.83) node [above] {$a$};
    \draw [black] (8,-10.6) -- (11.8,-10.6);
    \fill [black] (11.8,-10.6) -- (11,-10.1) -- (11,-11.1);
    \draw (9.9,-10.1) node [above] {$b$};
    \draw [black] (11.815,-18.161) arc (-102.22355:-158.8741:8.151);
    \fill [black] (11.81,-18.16) -- (11.14,-17.5) -- (10.93,-18.48);
    \draw (7.29,-16.88) node [below] {$c$};
    \draw [black] (17,-2.8) -- (20.6,-2.8);
    \fill [black] (20.6,-2.8) -- (19.8,-2.3) -- (19.8,-3.3);
    \draw (18.8,-2.3) node [above] {$c$};
    \draw [black] (25.8,-2.8) -- (29.1,-2.8);
    \fill [black] (29.1,-2.8) -- (28.3,-2.3) -- (28.3,-3.3);
    \draw (27.45,-2.3) node [above] {$b$};
    \draw [black] (17,-10.6) -- (20.6,-10.6);
    \fill [black] (20.6,-10.6) -- (19.8,-10.1) -- (19.8,-11.1);
    \draw (18.8,-10.1) node [above] {$a$};
    \draw [black] (17,-18.3) -- (20.6,-18.3);
    \fill [black] (20.6,-18.3) -- (19.8,-17.8) -- (19.8,-18.8);
    \draw (18.8,-17.8) node [above] {$b$};
    \draw [black] (25.8,-18.3) -- (29.1,-18.3);
    \fill [black] (29.1,-18.3) -- (28.3,-17.8) -- (28.3,-18.8);
    \draw (27.45,-17.8) node [above] {$b$};
    \end{tikzpicture}
    \end{center}

Добавим в полученный словарь слово $ab$. Для этого запишем слово на ленту и запустим автомат. Далее, когда автомат не сможет сделать переход, допишем недостающие состояния (префиксы) и добавим переходы так, чтобы в конце работы автомат смог принять новое слово. Т. е. для слова $ab$ перейдём в состояние <<$a$>>, а далее добавим новое принимающее состояние <<$ab$>> и переход в него по букве $b$:

\begin{center}
    \begin{tikzpicture}[scale=0.2]
    \tikzstyle{every node}+=[inner sep=0pt]
    \draw [black] (5.4,-18.2) circle (2.6);
    \draw (5.4,-18.2) node {$\epsilon$};
    \draw [black] (14.4,-10.4) circle (2.6);
    \draw (14.4,-10.4) node {$a$};
    \draw [black] (14.4,-18.2) circle (2.6);
    \draw (14.4,-18.2) node {$b$};
    \draw [black] (14.4,-18.2) circle (2);
    \draw [black] (14.4,-25.9) circle (2.6);
    \draw (14.4,-25.9) node {$c$};
    \draw [black] (14.4,-25.9) circle (2);
    \draw [black] (23.2,-10.4) circle (2.6);
    \draw (23.2,-10.4) node {$ac$};
    \draw [black] (23.2,-10.4) circle (2);
    \draw [black] (31.7,-10.4) circle (2.6);
    \draw (31.7,-10.4) node {$acb$};
    \draw [black] (31.7,-10.4) circle (2);
    \draw [black] (23.2,-18.2) circle (2.6);
    \draw (23.2,-18.2) node {$ba$};
    \draw [black] (23.2,-18.2) circle (2);
    \draw [black] (23.2,-25.9) circle (2.6);
    \draw (23.2,-25.9) node {$cb$};
    \draw [black] (31.7,-25.9) circle (2.6);
    \draw (31.7,-25.9) node {$cbb$};
    \draw [black] (31.7,-25.9) circle (2);
    \draw [black] (23.2,-2.8) circle (2.6);
    \draw (23.2,-2.8) node {$ab$};
    \draw [black] (23.2,-2.8) circle (2);
    \draw [black] (0.2,-18.2) -- (2.8,-18.2);
    \fill [black] (2.8,-18.2) -- (2,-17.7) -- (2,-18.7);
    \draw [black] (6.371,-15.795) arc (150.99991:110.82886:10.617);
    \fill [black] (11.88,-11.02) -- (10.96,-10.84) -- (11.31,-11.77);
    \draw (7.75,-12.43) node [above] {$a$};
    \draw [black] (8,-18.2) -- (11.8,-18.2);
    \fill [black] (11.8,-18.2) -- (11,-17.7) -- (11,-18.7);
    \draw (9.9,-17.7) node [above] {$b$};
    \draw [black] (11.815,-25.761) arc (-102.22355:-158.8741:8.151);
    \fill [black] (11.81,-25.76) -- (11.14,-25.1) -- (10.93,-26.08);
    \draw (7.29,-24.48) node [below] {$c$};
    \draw [black] (17,-10.4) -- (20.6,-10.4);
    \fill [black] (20.6,-10.4) -- (19.8,-9.9) -- (19.8,-10.9);
    \draw (18.8,-9.9) node [above] {$c$};
    \draw [black] (25.8,-10.4) -- (29.1,-10.4);
    \fill [black] (29.1,-10.4) -- (28.3,-9.9) -- (28.3,-10.9);
    \draw (27.45,-9.9) node [above] {$b$};
    \draw [black] (17,-18.2) -- (20.6,-18.2);
    \fill [black] (20.6,-18.2) -- (19.8,-17.7) -- (19.8,-18.7);
    \draw (18.8,-17.7) node [above] {$a$};
    \draw [black] (17,-25.9) -- (20.6,-25.9);
    \fill [black] (20.6,-25.9) -- (19.8,-25.4) -- (19.8,-26.4);
    \draw (18.8,-25.4) node [above] {$b$};
    \draw [black] (25.8,-25.9) -- (29.1,-25.9);
    \fill [black] (29.1,-25.9) -- (28.3,-25.4) -- (28.3,-26.4);
    \draw (27.45,-25.4) node [above] {$b$};
    \draw [black] (14.553,-7.82) arc (-194.08122:-264.28861:6.975);
    \fill [black] (20.62,-2.58) -- (19.78,-2.16) -- (19.88,-3.15);
    \draw (15.75,-3.75) node [above] {$b$};
    \end{tikzpicture}
    \end{center}

\newpage
Теперь удалим слово $ac$ из словаря. Для этого достаточно сделать его непринимающим:

\begin{center}
    \begin{tikzpicture}[scale=0.2]
    \tikzstyle{every node}+=[inner sep=0pt]
    \draw [black] (5.4,-18.2) circle (2.6);
    \draw (5.4,-18.2) node {$\epsilon$};
    \draw [black] (14.4,-10.4) circle (2.6);
    \draw (14.4,-10.4) node {$a$};
    \draw [black] (14.4,-18.2) circle (2.6);
    \draw (14.4,-18.2) node {$b$};
    \draw [black] (14.4,-18.2) circle (2);
    \draw [black] (14.4,-25.9) circle (2.6);
    \draw (14.4,-25.9) node {$c$};
    \draw [black] (14.4,-25.9) circle (2);
    \draw [black] (23.2,-10.4) circle (2.6);
    \draw (23.2,-10.4) node {$ac$};
    \draw [black] (31.7,-10.4) circle (2.6);
    \draw (31.7,-10.4) node {$acb$};
    \draw [black] (31.7,-10.4) circle (2);
    \draw [black] (23.2,-18.2) circle (2.6);
    \draw (23.2,-18.2) node {$ba$};
    \draw [black] (23.2,-18.2) circle (2);
    \draw [black] (23.2,-25.9) circle (2.6);
    \draw (23.2,-25.9) node {$cb$};
    \draw [black] (31.7,-25.9) circle (2.6);
    \draw (31.7,-25.9) node {$cbb$};
    \draw [black] (31.7,-25.9) circle (2);
    \draw [black] (23.2,-2.8) circle (2.6);
    \draw (23.2,-2.8) node {$ab$};
    \draw [black] (23.2,-2.8) circle (2);
    \draw [black] (0.2,-18.2) -- (2.8,-18.2);
    \fill [black] (2.8,-18.2) -- (2,-17.7) -- (2,-18.7);
    \draw [black] (6.371,-15.795) arc (150.99991:110.82886:10.617);
    \fill [black] (11.88,-11.02) -- (10.96,-10.84) -- (11.31,-11.77);
    \draw (7.75,-12.43) node [above] {$a$};
    \draw [black] (8,-18.2) -- (11.8,-18.2);
    \fill [black] (11.8,-18.2) -- (11,-17.7) -- (11,-18.7);
    \draw (9.9,-17.7) node [above] {$b$};
    \draw [black] (11.815,-25.761) arc (-102.22355:-158.8741:8.151);
    \fill [black] (11.81,-25.76) -- (11.14,-25.1) -- (10.93,-26.08);
    \draw (7.29,-24.48) node [below] {$c$};
    \draw [black] (17,-10.4) -- (20.6,-10.4);
    \fill [black] (20.6,-10.4) -- (19.8,-9.9) -- (19.8,-10.9);
    \draw (18.8,-9.9) node [above] {$c$};
    \draw [black] (25.8,-10.4) -- (29.1,-10.4);
    \fill [black] (29.1,-10.4) -- (28.3,-9.9) -- (28.3,-10.9);
    \draw (27.45,-9.9) node [above] {$b$};
    \draw [black] (17,-18.2) -- (20.6,-18.2);
    \fill [black] (20.6,-18.2) -- (19.8,-17.7) -- (19.8,-18.7);
    \draw (18.8,-17.7) node [above] {$a$};
    \draw [black] (17,-25.9) -- (20.6,-25.9);
    \fill [black] (20.6,-25.9) -- (19.8,-25.4) -- (19.8,-26.4);
    \draw (18.8,-25.4) node [above] {$b$};
    \draw [black] (25.8,-25.9) -- (29.1,-25.9);
    \fill [black] (29.1,-25.9) -- (28.3,-25.4) -- (28.3,-26.4);
    \draw (27.45,-25.4) node [above] {$b$};
    \draw [black] (14.553,-7.82) arc (-194.08122:-264.28861:6.975);
    \fill [black] (20.62,-2.58) -- (19.78,-2.16) -- (19.88,-3.15);
    \draw (15.75,-3.75) node [above] {$b$};
    \end{tikzpicture}
    \end{center} \qed

    \section*{Задача 3}

    Построить для словаря $S = \{ac, acb, b, ba, c, cbb\}$ автомат Ахо-Корасик. Посчитайте с его помощью количество различных вхождений слов из словаря $S$ в слово $acbacbb$ в качестве подслов.
    
    \bigskip
    
    \noindent \textbf{Решение.} \bigskip

    Построим автомат Ахо-Корасик:

    \begin{figure}[h!]
        \centering
            \begin{center}
        \includegraphics[width = 250pt]{image/picture3.png}
            \end{center}
    \end{figure}

    Теперь необходимо найти количество различных вхождений слов из словаря $S$ в слово $acbacbb$ в качестве подслов. Для этого воспользуемся следующим алгоритмом: запишем слово на ленту и запустим получившийся автомат Ахо-Корасика. Заведём счетчик $s$. Когда автомат попадает в принимающее состояние прямым переходом, увеличиваем счетчик на единицу: $s \leftarrow s+1$. Также будем проверять ссылки из очередного принимающего состояния, когда автомат будет переходить в это состояние: если по ссылкам можно добраться до какого-либо принимающего состояния, то увеличиваем счётчик на единицу (то состояние, куда пришли) и увеличиваем счётчик на число принимающих состояний, встретившихся на пути к подсчитанному состоянию. При этом, если перешли в принимающее состояние по ссылке, то этот переход не учитываем (он был учтен при прямом переходе). А также при подсчёте не учитываем и <<серые>> состояния, т.~к. эти суффиксы не содержатся в словаре.
    \newpage
    Конфигурации автомата приведены в таблице:

    \begin{table}[h!]
        \centering
        \begin{tabular}{|c|c|c|}
        \hline
        Состояние  & \begin{tabular}[c]{@{}c@{}}Необработанная \\ часть входа\end{tabular} & $+k$ \\ \hline
        $\epsilon$ & $acbacbb$                                                               & 0    \\ \hline
        $a$          & $cbacbb$                                                                & 0    \\ \hline
        $\textcolor{red}{ac}$         & $bacbb$                                                                 & 2   \\ \hline
        $\textcolor{red}{acb}$        & $acbb$                                                                  & 2   \\ \hline
        $cb$         & $acbb$                                                                  & 0!   \\ \hline
        $\textcolor{red}{b}$          & $acbb$                                                                  & 0!   \\ \hline
        $\textcolor{red}{ba}$         & $cbb$                                                                  & 1   \\ \hline
        $a$          & $cbb$                                                                   & 0!   \\ \hline
        $\textcolor{red}{ac}$         & $bb$                                                                    & 2   \\ \hline
        $\textcolor{red}{acb}$        & $b$                                                                     & 2   \\ \hline
        $cb$         & $b$                                                                     & 0!   \\ \hline
        $\textcolor{red}{cbb}$        & $\epsilon$                                                            & 2   \\ \hline
        \end{tabular}
        \end{table}

    Итак, с помощью алгоритма получили последовательность различных вхождений (11): $$ac, acb, c, b, ba, ac, acb, c, cbb, b, b.$$ 
    \qed

    \section*{Задача 4}
    
    Построить НКА, принимающий язык $L_3$, состоящий из слов в алфавите $\{a,b\}$, у которых третий от конца сивмол равен $a$. Затем, используя алгоритм, простроить эквивалентный полный ДКА.

    \bigskip

\noindent \textbf{Решение.} \bigskip

    Требуется построить автомат, принимающий язык, заданный РВ $R$: $$R = \Sigma^*a(a\;|\;b)(a\;|\;b).$$

    Это язык слов, содержащих суффиксы $aaa, aab, aba, abb$. \medskip

    Пусть $S = \{aaa, aab, aba, abb\}$ --- словарь. Тогда построим автомат Ахо-Корасик:

    \begin{figure}[h!]
        \centering
            \begin{center}
        \includegraphics[width = 200pt]{image/picture4.png}
            \end{center}
    \end{figure}

    \newpage
    Теперь построим эквивалентный полный ДКА:

    \begin{figure}[h!]
        \centering
            \begin{center}
        \includegraphics[width = 250pt]{image/picture5.png}
            \end{center}
    \end{figure} 
    \qed

\end{document}
