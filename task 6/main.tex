\documentclass[a4paper]{article}

\usepackage[T2A]{fontenc}
\usepackage[utf8]{inputenc}
\usepackage[russian]{babel}


\usepackage{graphicx}
\usepackage{float}
\usepackage{mathtools}
\usepackage{wrapfig}
\usepackage{amsfonts, amssymb, amsmath, latexsym}
\usepackage{nicefrac}
\usepackage{hhline}
\usepackage{multirow}
\usepackage[colorinlistoftodos,bordercolor=orange,backgroundcolor=orange!20,linecolor=orange,textsize=scriptsize]{todonotes}
\usepackage[colorlinks=true,linkcolor=blue,citecolor=blue]{hyperref}       % hyperlinks
\usepackage{nicefrac}       % compact symbols for 1/2, etc.
\usepackage{nameref}
\usepackage{booktabs}       % professional-quality tables

\usepackage{algorithm}
\usepackage{algpseudocode}

\usepackage{xcolor, colortbl}
\usepackage{etoolbox}

% \graphicspath{ {./} }

\usepackage[verbose=true,letterpaper]{geometry}

\newgeometry{
    textheight=9.5in,
    textwidth=6in,
    top=1in,
    headheight=12pt,
    headsep=25pt,
    footskip=30pt
}

\usepackage{epigraph}

%

\newcommand{\argmin}{\mathop{\arg\!\min}}
\newcommand{\argmax}{\mathop{\arg\!\max}}

\newcommand{\Var}{\mathbb{V}}
\newcommand{\Exp}{\mathbb{E}}
\newcommand{\Cov}{\text{Cov}}
\newcommand{\makebold}[1]{\boldsymbol{#1}}
\newcommand{\mean}[1]{\overline{#1}}
\newcommand{\eps}{\varepsilon}
\renewcommand{\epsilon}{\varepsilon}

\newcommand{\partfrac}[2]{\frac{\partial #1}{\partial #2}}
\newcommand{\ttt}[1]{\texttt{#1}}
\newcommand{\term}[1]{\textbf{#1}}

\newcommand{\la}{\langle}
\newcommand{\ra}{\rangle}

\newcommand{\lp}{\left(}
\newcommand{\rp}{\right)}
\newcommand{\lf}{\left\{}
\newcommand{\rf}{\right\}}
\newcommand{\ls}{\left[}
\newcommand{\rs}{\right]}
\newcommand{\lv}{\left|}
\newcommand{\rv}{\right|}

\newcommand*{\affaddr}[1]{#1} % No op here. Customize it for different styles.
\newcommand*{\affmark}[1][*]{\textsuperscript{#1}}


\usepackage{subcaption}
%\usepackage[font={small}]{caption}

\usepackage{amsthm}
\usepackage{tikz}

\theoremstyle{definition}
\newtheorem{definition}{Определение}[section]

\newtheorem{exercise}{Задача}[section]

\newtheorem*{solution}{Решение}
\theoremstyle{remark}
\newtheorem*{remark}{Remark}

\makeatletter
\renewcommand{\l@section}{\@dottedtocline{1}{0em}{2.1em}}
\makeatother

% \setlength\epigraphwidth{.8\textwidth}
\setlength\epigraphrule{0pt}

\title{ТРЯП. Домашнее задание № 6}
\author{Шарапов Денис, Б05-005}
\date{}

\begin{document}

\maketitle

\section*{Задача 1}

Будет ли регулярным язык $L_3$ всех слов в алфавите $\{0,1\}$, которые представляют числа в двоичной записи, дающие остаток два при делении на три?

\bigskip

\noindent \textbf{Решение.} \bigskip

Зададим отношение $L_3$-эквивалентности на языке всех слов алфавита $\Sigma = \{0,1\}$. Разобъем~$\Sigma^*$ на классы эквивалентности $L_0, L_1, L_2$, где 

\begin{itemize}
    \item[] $L_0$ --- язык всех слов, дающих остаток 0 при делении на 3,
    \item[] $L_1$ --- язык всех слов, дающих остаток 1 при делении на 3,
    \item[] $L_2$ --- язык всех слов, дающих остаток 2 при делении на 3.
\end{itemize}

Проверим, являются ли эти множества классами эквивалентности по остатку при делении на 3. 

\begin{enumerate}
    \item $\forall i \; \forall v, u \in L_i \hookrightarrow u \sim_{L_3} v$.
    \item Пусть слова $u \in L_i$ и $v \in L_j$ принадлежат разным классам эквивалентности $L_i$ и $L_j$. Тогда к словам $u$ и $v$ припишем справа некоторое слово $z \in \Sigma^*$, которое имеет остаток~$q$ при делении на 3. Получим, что остатки у получившихся слов отличаются, значит они принадлежат разным классам эквивалентности.
    \item Возможные остатки при делении на 3: 0, 1 или 2. Т. е. выбранные множества покрывают~$\Sigma^*$.
\end{enumerate}

Получили, что у $L_3$ конечное число классов эквивалентности, что по теореме Майхилла-Нероуда равносильно его регулярности. \qed

\section*{Задача 2}

Описать классы эквивалентности Майхилла-Нероуда для языка $L$ над алфавитом $\Sigma = \{a,b\}$. В случае конечности множества классов построить минимальный полный ДКА, распознающий~$L$, где $L$ --- язык

\begin{itemize}
    \item[a)] $\Sigma^*ab\Sigma^*$;
    \item[б)] $\text{PAL} = \{w : w = w^{R}\}$;
    \item[в)] $\{w : |w|_{ab} = |w|_{ba}\}$.   
\end{itemize}
 
\noindent \textbf{Решение.}

\begin{itemize}
    \item[a)]  Разобъём множество $\Sigma^*$ на три множества: 
        \begin{enumerate}
            \item $L_1 = \{xaby : x,y \in \Sigma^*\}$,
            \item $L_2 = \{w : w \in b^*\}$,
            \item $L_3 = \{xay : x \in b^*, y \in a^*\}$.
        \end{enumerate} 

        Слова из $L_2$ $L$-эквивалентны друг другу, т. к. какое слово бы не приписали, они одновременно либо лежат, либо не лежат в языке $L$ (их одновременная принадлежность зависит от приписываемого слова). 

        Слова из $L_3$ $L$-эквивалентны друг другу, т. к. по построению все слова из этого языка заканчиваются на $a$, и их одновременная принадлежность записит только от приписываемого слова.

        Пусть $z = a$ --- различающее слово. Припишем его к трём произвольным словам из $L_1, L_2, L_3$. Тогда слова из $L_2$ и $L_3$ не принадлежат языку $L$, а слово из $L_1$ принадлежит языку $L$.

        Пусть $z = b$ --- различающее слово. Припишем его к произвольным словам из $L_2, L_3$. Тогда слово из $L_2$ не будет принадлежать языку $L$, а слово из $L_3$ будет принадлежать языку $L$.
        
        Таким образом показано, что произвольная пара слов, взятая из разных множеств $L_1$, $L_2$, $L_3$, не является $L$-эквивалентой.

        Покажем, что выбранные множества покрывают $\Sigma^*$. Пусть $w \in \Sigma^*$ --- произвольное слово из множества всех слов над алфавитом $\Sigma$. Тогда в нём либо есть $ab$ в качестве подслова, либо в нем нет $ab$ в качестве подслова. Если есть, то оно попадает во множество $L_1$. Если нет, то в нём есть слово $a$ в качестве подслова, и тогда оно попадает во множество~$L_3$. Если в нём нет слова $a$ в качестве подслова, то оно попадает во множество $L_2$. 
        
        Т. е. было показано, что произвольное слово $w \in \Sigma^*$ попадает в объединение $$w \in \bigcup \limits_i L_i = \Sigma^*.$$

        Теперь построим полный ДКА:

        \begin{figure}[h!]
            \centering
                \begin{center}
            \includegraphics[width = 150pt]{image/picture2.1.png}
                \end{center}
        \end{figure}
    \item[б) ] 
        Покажем, что классов $L$-эквивалентности бесконечно много. Рассмотрим некоторое множество $L_i$, которое может являться классом $L$-эквивалентности. Рассмотрим теперь два произвольных слова $u \in L_i, v \in L_i$. Они являются $L$-эквивалентными, т. к., если к ним приписать справа одно и то же слово, то они оба будут одновременно либо принадлежать~$L_i$, либо не принадлежать $L_i$.
        
        Рассмотрим два различных множества $L_i$ и $L_j$. Пусть $u \in L_i$, $v \in L_j$. Тогда, если длины этих слов совпадают, то за разделяющее слово $z \in \Sigma^*$ достаточно взять $z = u^R$. Если длины не совпадают, то рассматриваем слова $u$ и $v$ следующим образом. Допустим, что на $l+1$-ом месте слова $u$ стоит буква $b$. Тогда достаточно рассмотреть разделяющее слово $z = au^R$, т. к. $uau^R \in L$, а $vau^R \notin L$ (на $l$-ом месте при чтении с конца и сначала стоят разные буквы).

        Получившиеся множества $L_i$ покрывают всё множество $\Sigma^*$, т. к. каждое слово над алфавитом $\Sigma$ попадает в свой класс $L$-эквивалентности.

        Таким образом получили бесконечно много классов эквивалентности, что по теореме Майхилла-Нероуда равносильно нерегулярности языка $L$.
    \item[в) ] 
        Рассмотрим множества:
        \begin{itemize}
            \item[] $A_i = \{w = w_0w_1 \ldots w_k \ldots a : |w|_{ab}-|w|_{ba} = i\}$,
            \item[] $B_i = \{w = w_0w_1 \ldots w_k \ldots b : |w|_{ab}-|w|_{ba} = i\}$,
            \item[] $C = \{\varepsilon\}$. 
        \end{itemize}

        Покажем, что два произвольно выбранных слова из одного множества $L$-эквиваленты. Действительно, разность у слов одинаковая, и приписывание произвольного слова $z~\in~\Sigma^*$ изменит эту разность одинаково (т. е. слова будут либо одновременно принадлежать языку~$L$, либо не принадлежать языку $L$).
        
        Покажем, что два произвольно выбранных слова из разных множеств не являются $L$-эквивалентыми. Если оба слова выбраны одновременно из одной последовательности $A_i$ или $B_i$, то приписывание справа произвольного слова $z \in \Sigma^*$ изменит количество $ba$ и $ab$ на одно и то же число, и разности останутся неравными. Если одно слово выбрано из $A_i$, а другое выбрано из $B_i$, то приписывание справа произвольного слова $z \in \Sigma^*$ в одном слове изменит разность, а в другом --- нет (конструктивно: если слово заканчивается на~$a$, то разделяющим словом берём $z = bw, |w|_{ab} = |v|_{ab} - 1, |w|_{ba} = |v|_{ba},$ где $v \in A_i$).

        Выбранные множества покрывают $\Sigma^*$, так как они выбраны на множестве всех слов.

        Таким образом получили бесконечно много классов эквивалентности, что по теореме Майхилла-Нероуда равносильно нерегулярности языка $L$. \qed
\end{itemize}

\section*{Задача 3}

Язык $L$ состоит из двоичных записей (без ведущих нулей) положительных чисел $n$, входящих в пару $(n,m)$ некоторого решения уравнения $5n+3m = 17$ в целых числах. Описать классы эквивалентности Майхилла-Нероуда языка $L$. Является ли язык $L$ регулярным?

\bigskip

\noindent \textbf{Решение.} \bigskip

С помощью расширенного алгоритма Евклида получим решение уравнения: $$n = 3k + 1.$$ Аналогично задаче 1 найдём классы $L$-эквивалентности и получим, что язык $L$ --- регулярный, где $L$ --- язык всех слов, дающих в остатке 1 при делении на 3. 

\medskip Классы $L$-эквивалентности: 

\begin{itemize}
    \item[] $L_0$ --- язык всех слов, дающих остаток 0 при делении на 3,
    \item[] $L_1$ --- язык всех слов, дающих остаток 1 при делении на 3,
    \item[] $L_2$ --- язык всех слов, дающих остаток 2 при делении на 3.
\end{itemize}


\bigskip
    
\noindent  \textbf{Ответ:} Да, является регулярным.

\section*{Задача 4}

Являются ли регулярными следующие языки:

\begin{itemize}
    \item[a) ] $\{xy : |x| > |y|, \; \text{$x$ содержит букву $a$}\}$,
    \item[б) ] $\{xy : |x| < |y|, \; \text{$x$ содержит букву $b$}\}$?
\end{itemize}

\noindent \textbf{Решение.}

\begin{itemize}
    \item [a) ] Пусть $$ L = \{xy : |x| > |y|, \; \text{$x$ содержит букву $a$}\}.$$ Выберем два множества: $L_1$ --- язык всех слов, содержащих букву $a$, $L_2$ --- язык всех слов, не содержащих букву $a$. Проверим, являются ли они классами $L$-эквивалентности.
    
    Если взять два произвольных слова из одного множества $L_1$ или $L_2$, то приписывание произвольного слова $z\in \Sigma^*$ меняет принадлежность к $L$ получившихся слов одновременно (в случае $L_1$ можно взять, например, $y = \varepsilon$).

    Пусть теперь $u \in L_1, v \in L_2$ и $z = b$ --- разделяющее слово. Видно, что при приписывании справа слова $z$ к словам $u$ и $v$ получаем, что $uz \in L$, а $vz \notin L$.

    Заданные множества покрывают $\Sigma^*$.

    Таким образом получили два класса $L$-эквивалентности, что по теореме Майхилла-Нероу-да равносильно регулярности языка $L$. \qed

\end{itemize}

\section*{Задача 5} Левым языком $L_q$ для состояния $q$ автомата назовём множество $$L_q = \{x : q_0 \stackrel{x}{\rightarrow} q \}.$$
Пусть $\mathcal{A}$ --- полный ДКА, распознающий язык $L$. Доказать, что 

\begin{itemize}
    \item[a) ] Каждый левый язык $L_q$ является подмножеством некоторого класса $L$-эквивалентности.
    \item[б) ] Для каждого класса эквивалнтности $[x]$ существует такое подмножество состояний \\ $Q_x \subseteq~Q_{\mathcal{A}}$, что $$[x] = \bigcup \limits_{q\in Q_x} L_q.$$
    \item[в) ] Если $x \in L_q$, то $$L_p \subseteq [x] \Longleftrightarrow R_q = R_p.$$
\end{itemize}

\noindent \textbf{Решение.}

\begin{itemize}
    \item[a) ] Требуется доказать, что слова из $L_q$ лежат в одном классе $L$-эквивалентности. Действительно, рассмотрим два произвольных слова из $L_q$. Автомат $\mathcal{A}$ --- полный ДКА, поэтому количество состояний конечно и в нём определены переходы по всем буквам алфавита~$\Sigma$. Припишем к двум произвольным словам из $L_q$ произвольное слово $z\in\Sigma^*$. Запустим автомат. Через некоторое время работы автомат придёт в состояние $q$ по определению множества~$L_q$ для обоих слов. Далее он будет обрабатывать слово $z$ для двух слов из~$L_q$, поэтому он одновременно для них обоих придёт либо в принимающее состояние, либо в непринимающее состояние. Т. е. все слова из $L_q$ принадлежат некоторому классу $L$-эквивалентности. \qed
    \item[б)] Рассмотрим все вершины автомата $\mathcal{A}$ (их конечное множество). Для каждой вершины рассмотрим множество $L_q$. Далее будем рассматривать $L$-эквивалентность слов из двух разных $L_q$. Если окажется, что некоторые слова являются эквивалентными, то в дальнейшем будем рассматривать эти два языка вместе (должна выполняться транзитивность). Таким образом получится некоторое разбиение множества всех $L_q$ для каждой вершины автомата на множества $L_{q_i}$ (которые попарно неэквивалентны). Т. е. существует такое подмножество состояний, что каждому классу эквивалентности соответсвует язык $L_q$, который представим в виде объединения $L_{q_i}$ по всем состояниям из рассматриваемого подмножества состояний. \qed
    \item[в) ] Если слова из $L_p$ и $L_q$ лежат в одном классе эквивалентности, то приписывание к ним справа произвольного слова $z \in \Sigma^*$ равносильно тому, что автомат закончит работу в одном и том же состоянии. Если при этом известно, что автомат закончил работу в принимающем состоянии, то это эквивалентно условию $R_q = R_p$ (в противном случае было бы противоречие принадлежности одному классу эквивалентности). \qed
\end{itemize}

\newpage

\section*{Задача 6}

Построить минимальный ДКА по диаграмме.

    \bigskip

\noindent \textbf{Решение.} 

Воспользуемся алгоритмом минимазации ДКА. C помощью вершины $D$ дополним ДКА, заданный диаграммой: 

\begin{figure}[h!]
    \centering
        \begin{center}
    \includegraphics[width = 200pt]{image/picture5.1.png}
        \end{center}
\end{figure}

Далее воспользуемся алгоритмом:

\begin{figure}[h!]
    \centering
        \begin{center}
    \includegraphics[width = 200pt]{image/picture5.2.png}
        \end{center}
\end{figure}

Откуда построим минимальный ДКА:

\begin{figure}[h!]
    \centering
        \begin{center}
    \includegraphics[width = 120pt]{image/picture5.3.png}
        \end{center}
\end{figure} \qed

\section*{Задача 7}

Показать, что следующий язык удовлетворяет лемме о разрастании для регулярных языков, но сам регулярным не является:

$$L = \{ ab^{2^i} : i \geq 0 \} \cup \{ b^j : j \geq 0\ \} \cup \{ a^mb^n : m > 1, n \geq 0\}.$$

\newpage

\noindent \textbf{Решение.} \bigskip

Пусть 

\begin{itemize}
    \centering
    \item[] $L_1 = \{ ab^{2^i} : i \geq 0 \}$,
    \item[] $L_2 = \{ b^j : j \geq 0\ \}$,
    \item[] $L_3 = \{ a^mb^n : m > 1, n \geq 0\}$. 
\end{itemize}

Покажем, что выполняется лемма о накачке для языка $L = L_1 \cup L_2 \cup L_3$.
\medskip
Пусть $p =~1, x =~\varepsilon, \\ y = a, z = b^{2i}$ для $i \geq 0$. Тогда:

\begin{itemize}
    \centering
    \item[] при $k = 0$ выполнено $w = b^{2^i} \in L_2 \subseteq L$,
    \item[] при $k = 1$ выполнено $w = ab^{2^i} \in L_1 \subseteq L$,
    \item[] при $k > 0$ выполнено $w = a^kb^{2^i} \in L_3 \subseteq L$.  
\end{itemize}

Аналогично даказываем выполнимость леммы о накачке для слов, взятых из $L_2$ $(p = 1$, $x = \varepsilon,$ $y = b)$. \medskip

Для языка $L_3$ построим автомат ДКА, распознающий его (корректность доказывается по индукции числа букв $a$ и $b$ в слове):

\begin{figure}[h!]
    \centering
        \begin{center}
    \includegraphics[width = 180pt]{image/picture7.1.png}
        \end{center}
\end{figure}

Теперь покажем, что $L$ не является регулярным. От противного. Предположим, что язык~$L$ является регулярным: $L \in \text{REG}$. Регулярные языки замкнуты относительно разности (дополнения) и объединения. Тогда верно следующее: $$L_1 \cap L_2 = \varnothing, \; L_2 \cap L_3 = \varnothing \; L_1 \cap L_3 = \varnothing \Longrightarrow L_1 = L \;\backslash\; (L_2 \cup L_3).$$ Докажем, что $L_1 \notin \text{REG}$. \medskip

Рассмотрим слово $w = ab^{2^p}$, где $p$ --- константа леммы. Тогда для него существует разбиение (при котором выполняются условия леммы) такое, что $$x = a, \; y = b^{{k-1}}, \; z = b^{{k+1}}, \; k = 2^p.$$ Eсли $L_1$ был бы регулярным, то условие $xy^iz \in L_1$  выполнялось для любого $i$. Но в данном случае условие выполняется в зависимости от чётности $i$ (для любого $p \geq 1$ сможем подобрать такое $i$, что принадлежность не выполняется). Т. е. выполняется отрицание леммы о накачке.

\medskip

Таким образом показали, что для языка $L$ выполняется лемма о накачке, но он не является регулярным в силу нерегулярности $L_1$ и регулярности $L_2$ и $L_3$. \qed

\end{document}
